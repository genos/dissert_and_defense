\bodychapter{\texttt{e2c2}: A C++11 library for Edwards ECC}
\label{chp:e2c2}

In order to explore the theory and implementation of Edwards curves for
    Elliptic Curve Cryptography, I've created a modern C++11 software library
    called \texttt{e2c2}.
In this chapter, we'll discuss the design and rationale behind \texttt{e2c2},
    its organization, and a few examples of its functionality.


\bodysection{Rationale and Design Choices}

\texttt{e2c2} was designed to be a software library that was simple enough to
    use but close enough to a practical implementation that software engineers
    could use it as a template or a stand-in for a practical cryptographic
    library.
As such, it strikes a balance between usability and speed and between clarity
    of exposition and being closer to ``production-level code.''

\bodysubsection{C++11}

Probably the most obvious design choice was picking C++, specifically the
    newest standard C++11, for the programming language.
Though not as low-level as C, C++ still gives us decent enough control over
    the underlying specifics of the machine.
Since it's a compiled language with a long history of implementation, it
    creates very fast binaries.
Furthermore, C++ is available on a number of platforms, so portability is less
    of an issue.
These characteristics combine to make C++ a viable choice for a cryptography
    implementation seeking to bridge the gap between mathematical theory and
    programming practice.

Even more helpful was the expressiveness of the new C++ language standard,
    C++11\cite{jtcsc22}.
This new standard adds interesting and useful constructs to the language, many
    of which are used in our library.
These improvements---like type inference, lambda functions and expressions, and
    range-based for-loops---make C++ an easier language in which to implement
    complex mathematical and cryptographic concepts.
What's more, this new level of expressiveness comes at no obvious loss of
    execution speed.

\bodysubsection{NTL and GMP}

The next important choice was choosing to use software libraries that abstract
    away the details of arbitrary precision integers and foundational number
    theory concepts.
The GNU Multiple Precision Arithmetic Library, better known as GNU MP or GMP,
    is a free and open-source library for arbitrary-precision arithmetic with
    integers and rational numbers.\cite{granlund1996gnu}
Though a true full-blown production-level cryptographic library might choose to
    implement its own big integer arithmetic for ultimate control, speed, and
    safety, GMP is tried and tested enough after over 15 years of development
    to be included here in our library that's trying to be a mix of
    proof-of-concept and implementation guide.

I used Victor Shoup's Number Theory Library---NTL for short---for the same
    reasons, only more so.\cite{shoup2009ntl}
NTL is a C++ library for doing number theory; to quote its website,
\begin{quote}
NTL is a high-performance, portable C++ library providing data structures and
    algorithms for manipulating signed, arbitrary length integers, and for
    vectors, matrices, and polynomials over the integers and over finite
    fields.
NTL provides high quality implementations of state-of-the-art algorithms for:
\begin{itemize}
\item arbitrary length integer arithmetic and arbitrary precision floating
    point arithmetic;
\item polynomial arithmetic over the integers and finite fields including basic
    arithmetic, polynomial factorization, irreducibility testing, computation
    of minimal polynomials, traces, norms, and more;
\item lattice basis reduction, including very robust and fast implementations
    of Schnorr-Euchner, block Korkin-Zolotarev reduction, and the new
    Schnorr-Horner pruning heuristic for block Korkin-Zolotarev;
\item basic linear algebra over the integers, finite fields, and arbitrary
    precision floating point numbers.
\end{itemize}
NTL's polynomial arithmetic is one of the fastest available anywhere, and has
    been used to set ``world records'' for polynomial factorization and
    determining orders of elliptic curves.
\end{quote}
NTL does most of the heavy lifting when it comes to finite field computations
    in \texttt{e2c2}; it is only linked against GMP at compile time to take
    advantage of some of its code for enhanced performance.
To quote the project's website again, the main reason for choosing NTL is
    because ``It provides a good environment for easily and quickly
    implementing new number-theoretic algorithms, without sacrificing
    performance.''

In some regards, \texttt{e2c2} can be seen as an extension of NTL.
It uses portions of NTL to implement Edwards curves in a way that blends well
    with the rest of NTL, and the line between where NTL stops and
    \texttt{e2c2} starts need not be of much concern to users of this
    library.\footnote{Especially if C++ projects using \texttt{e2c2} prefaces
    any major portions of code with the directives
    ``\texttt{using namespace NTL}'' and ``\texttt{using namespace e2c2},'' as
    in the examples we'll show later.}

\bodysubsection{Template Specialization Instead of Inheritance}

The last design choice we'll discuss will probably only be of concern to
    aficionados of C++ (or perhaps another object-oriented language).
Though curves and points are implemented as classes in \texttt{e2c2}, the
    library is written using template specialization instead of class
    inheritance.
This gave me the ability to use the illusion of inheritance in constructing
    \texttt{e2c2}, sharing common functionality between different types of
    curves or different types of points, without the added runtime penalties
    that can be associated with virtual function lookup.
Moreover, C++ templates are expanded at compile time (and much can be inlined
    by a compiler), thereby only charging the programmer for what they use.
Another added bonus of using template specialization over inheritance is ease
    of portability and use; \texttt{e2c2} consists entirely of C++ header
    files, four specifying the implementation and one single \texttt{e2c2.h}
    interface header to be included in projects.
This means that \texttt{e2c2} is quite compact, and it is simple to share or
    extend the sourcecode.
It's compact enough that the entire source for \texttt{e2c2} and examples of
    its usage is included in Appendix \ref{chp:src}.

Readers interested in learning more about C++ templates should consult
    \cite{stroustrup1997c++}.

\bodysection{Curves and Points}

We now get into the actual details of \texttt{e2c2}'s code.
First up is a description of its fundamental objects: curves and points.

\bodysubsection{\texttt{curves.h}}

The code for curves in contained in the header file \texttt{curves.h}.
Curves are implemented as templates based on two parameters: an element type
    (one of NTL's ``ZZ\_pE'' or ``GF2E'' datatypes) and a C++
    enumeration\footnote{Actually it's a C++11 strongly typed enumeration for
    added typesafety peace of mind.} called ``CurveID'' (to help distinguish
    between two types of curves that have the same base element type).
Each curve type has two field elements called $c$ and $d$ as members; these
    correspond to the $c$ and $d$ of Edwards curves, $a$ and $d$ in twisted
    Edwards curves, and $d_1$ and $d_2$ in binary Edwards curves.
They also have an NTL ZZ $m$ that gives the cardinality of the curve (i.e., the
    number of rational points on the curve over the field in question).

Besides the usual C++ member functions (i.e. constructors and destructors),
    curves have four member functions.
Through C++'s operator overloading, curves can be compared for equality or
    inequality with the usual operators (\texttt{==} and \texttt{!=}).
They can also tell you their cardinality via calls to \texttt{cardinality()}
    and information about them can be printed to an output stream via the
    \texttt{<<} operator.

Each curve specialization---OddCurve, TwistedCurve, and BinaryCurve---provides
    four things.
The first is an alias typedef from the verbose template name to the shorted
    curve name for ease of use.
Then each specialization provides three functions that are used in their member
    functions:
\begin{itemize}
\item   \texttt{getName}, when given a curve as an argument, returns a human
    readable string describing the type of curve; this is used when the curve
    information is printed to an output stream
\item   \texttt{parametersValid}, when given a curve as an argument, returns a
    boolean value which checks whether the curve parameters $c$ and $d$ are
    valid; this is used in curve construction to ensure that the curve being
    built matches the requirements given in the various papers about it
\item   \texttt{curveEquation}, when given a curve and an $x$ and $y$
    coordinate, returns a boolean value that specifies whether the two
    coordinates describe a point lying on the curve or not; this is used when
    points are constructed (to be described shortly)
\end{itemize}

To construct a curve, one first builds the appropriate field via NTL, then
    calls the specific Curve constructor with $c$, $d$, and $m$ specified.
For an example, see \texttt{curves\_test.cc} (described below).

\bodysubsection{\texttt{points.h}}

As you might expect, points are a little more complicated.
There are two base class templates for points: ``Affine'' and ``Projective.''
Each point has a curve to which it is assigned; in addition, Affine points have
    two coordinates ($x$ and $y$), while Projective points have three
    coordinates ($x$, $y$, and $z$).
Though Affine points can only interact with Affine points at this time, and
    likewise Projective, there is a copy constructor from Affine to Projective;
    i.e. if \texttt{a} is a BinaryAff, one can write the following:
\begin{lstlisting}
auto p = BinaryProj(a);
\end{lstlisting}
    to create a projective version of \texttt{a}.\footnote{This copy
    constructor has been declared \texttt{explicit}, so there shouldn't be any
    surprises about when this conversion takes place.}

Beyond the obvious difference in the number of coordinates, Affine and
    Projective points have all the same member functions. 
Points can be compared for equality and inequality, tested for whether they're
    the identity element, added or subtracted (both with \texttt{+}, \texttt{-}
    and \texttt{+=}, \texttt{-=}), doubled, and multiplied by a scalar.
Scalar multiplication is implemented with Montgomery's ladder, and incorporates
    the suggested ``wrap-around'' to counteract Brumley and Tuveri's timing
    attack from \cite{brumley2011timing}.

An example of point functionality is given in \texttt{points\_test.cc}, which
    we will soon discuss.

\bodysection{Utilities and Subroutines}

\texttt{e2c2} has two scaffolding files that support the real work, the first
    of which is the header \texttt{mol.h}.

\bodysubsection{\texttt{mol.h}}

This header file is named for the authors of \cite{moloneyefficient}.
In it there are a number of functions and routines that implement the ideas
    from that paper.
Some are only used by other functions in the same file, but probably most
    important is \texttt{mol\_alg\_1}.
If fed a long $n$, representing the degree of the extension field
    $\mathbb{F}_{2^n}$, and two elements of this field $a_2$ and $a_6$ that
    specify a binary elliptic curve in Weierstrass form, this algorithm
    computes the $d_1$ value that specifies the birationally equivalent binary
    Edwards curve.\footnote{$d_1$ is called $c$ in \texttt{e2c2}.}
This function is in turn used to implement other functions in \texttt{mol.h}.

\texttt{mol\_alg\_1} is definitely too low-level for typical use, while other
functions in this file are probably more user friendly; programmers interested
    in the \texttt{e2c2} library will find most useful
\begin{itemize}
\item \texttt{from\_weierstrass}, which takes $n$, $m$, $a_2$, and $a_6$ as
    arguments and returns a BinaryCurve
\item \texttt{mol\_bm\_aff}, which implements the birational map from
    \cite{moloneyefficient} for Affine points
\item \texttt{mol\_bm\_proj}, which implements the birational map from
    \cite{moloneyefficient} for Projective points
\end{itemize}

\bodysubsection{\texttt{utilities.h}}

The other utility file is called, rather aptly, \texttt{utilities.h}.
This file contains a routine to set a parameter given a string in hex or not;
    this routine, called \texttt{set\_parameter}, is helpful in constructing
    the curves specified in various standards like \cite{gallagher09fipspub}.
\texttt{utilities.h} also specifies the various C++ exceptions created to
    signal fatal error conditions to users of \texttt{e2c2}:
\begin{itemize}
\item \texttt{InvalidParametersException}, which is thrown when one tries to
    construct a curve with parameters that are invalid
\item \texttt{DifferentCurvesException}, which is thrown when attempting to
    perform an operation involving two points from different curves
\item \texttt{NotImplementedException}, which is left over from \texttt{e2c2}'s
    development; it was used (and can be used again, as development of this
    library continues) to indicate that the edge of current implementation had
    been reached but the functionality in question was planned for future work
\end{itemize}

\bodysection{Examples}

Probably the best exposition we can have of \texttt{e2c2} is to see it in
    action; as such, we present four example C++ files that can be compiled
    and run to demonstrate the library's functionality and usage.
All examples have been compiled and tested with the following command and
    options:\footnote{The \texttt{-Weffc++} option tells the compiler to warn
    about C++ code that doesn't meet the high standards set by
    \cite{meyers2005effective}.}
\begin{lstlisting}[caption={Compiler Options},language=HTML]
g++-4.7 -Wall -Wextra -Weffc++ -pedantic -O3 -m64 -std=c++11 -lntl -lgmp
\end{lstlisting}
Specifically, all this code was compiled with the GNU C++ compiler version 4.7
    (see \cite{stallman2002gnu}), though any compiler and standard library that
    implements most of the C++11 standard should work.

\bodysubsection{\texttt{curves\_test.cc}}

The first example file is \texttt{curves\_test.cc}.
In this program, there are three different subroutines that can all be called
    by the main routine, depending on user input.
Each specifies a type of Edwards curve---odd, binary, or twisted---to construct
    and output some information about.
The binary example is slightly more involved, since it builds a curve from the
    Weierstrass equivalent and later intentionally crashes by trying to make a
    curve with invalid parameters.

\bodysubsection{\texttt{points\_test.cc}}

The next example file is similar; it shows the creation and usage of points,
    both affine and projective, over all three types of Edwards curves.
Again, the user can specify a specific type of curve to work over; once that is
    done, the program
\begin{enumerate}
\item builds the appropriate curve
\item constructs an affine identity element on that curve and outputs it
\item constructs two projective points, one of which is the neutral element and
    one which is not
\item demonstrates point addition, negation, and scalar multiplication with
    these points
\end{enumerate}

\bodysubsection{\texttt{key\_demo.cc}}

The third example file is a little more involved.
This program gives a short Diffie-Hellman key exchange demonstration.
After some setup, we join our friends Alice and Bob as they try to construct a
    shared key so as to communicate in private.
After deciding (in public) on a base point $P$, Alice and Bob pick private
    random scalars $a$ and $b$, respectively.
Then Alice computes and publishes $aP$ while Bob does the same with $bP$.
Their shared key is $a(bP) = b(aP)$; the code double-checks that all
    calculations went according to plan.

Here is the output of an example run of this program, slightly reformatted:
\lstinputlisting[caption={Output of key\_demo.cc},language=HTML]{listings/key_demo.txt}

\bodysubsection{\texttt{ecoh\_echo.cc}}

As a final example, we draw the reader's attention to the sourcecode listing in
    the first section of Chapter \ref{chp:app}.
In it we provide \texttt{ecoh\_echo.cc}, a reference implementation of our
    password based key derivation function described in that section.
It is probably the best non-trivial example of using \texttt{e2c2} for
    cryptographic exploration and implementation.
