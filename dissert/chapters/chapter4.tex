\bodychapter{Practical Considerations}
\label{chp:prax}

Binary Edwards curves specifically, and Edwards curves in general, have
    generated a lot of excitement in the cryptographic field.
As such, a number variations, adaptations, and entirely new normal forms have
    been proposed in recent years.
Many of them are promising and have interesting mathematical properties; that
    doesn't mean, unfortunately, that they are ``ready for primetime'' as far
    as cryptographic implementation is concerned.
In this chapter, we show that four new normal forms for elliptic curves,
    despite being mathematically interesting and involving some quite nice
    theory, do not measure up to the cryptographic standard set by binary
    Edwards curves.\footnote{A previous version of this chapter has been posted
    to the International Association for Cryptologic Research's cryptology
    eprint archive (\texttt{http://eprint.iacr.org/2013/015}) and has been
    submitted to IACR's CRYPTO 2013 conference
    (\texttt{http://www.iacr.org/conferences/crypto2013/}).}
As we shall see, these constructions exhibit weaknesses that fall into one of
    two categories: either their group law is not symmetric, so commutativity
    is hard to see (though of course still present), or their atypical choice
    of neutral point obfuscates the result of adding a point and the neutral
    element.
In both cases, one has to resort to working modulo the curve equation (or more
    precisely, modulo the ideal generated by the curve equation in the
    appropriate polynomial ring) to see that these computations behave as
    expected.
This means that elementary operations, the results of which should be
    immediately apparent, cannot be implemented programmatically in a simple
    way; even simple work must involve unnecessary checks and reductions.
This extra work will at best slow down a cryptosystem, and at worst could leak
    enough side-channel information to severely weaken the system.

\bodysection{Two Weaknesses \& How Edwards Curves Avoid Them}

Recall that Edwards curves, originally presented by Edwards in
    \cite{edwards2007normal} and expanded upon by Bernstein and Lange in
    \cite{bernstein2007faster}, are elliptic curves over a field of
    characteristic not equal to two of the form
\[
x^2 + y^2 = c^2(1 + dx^2y^2)
\]
    with some restrictions on $c$ and $d$ and have the affine group law
\begin{equation}\label{eq:edwards_add}
(x_1, y_1) + (x_2, y_2) =
\left(
\frac{x_1y_2 + y_1x_2}{c(1 + dx_1x_2y_1y_2)},
\frac{y_1y_2 - x_1x_2}{c(1 - dx_1x_2y_1y_2)}
\right)
\end{equation}
Next, twisted Edwards curves can be taken over any non-binary field, have the
    form
\[
ax^2 + y^2 = 1 + dx^2y^2
\]
    and have affine group law
\begin{equation}\label{eq:twisted_add}
(x_1, y_1) + (x_2, y_2) =
\left(
\frac{x_1y_2 + y_1x_2}{1 + dx_1x_2y_1y_2},
\frac{y_1y_2 - ax_1x_2}{1 - dx_1x_2y_1y_2}
\right)
\end{equation}
Finally, binary Edwards curves take the form
\[
d_1(x + y) + d_2(x^2 + y^2) = (x + x^2)(y + y^2)
\]
    over a field of characteristic two, and have the (slightly more complicated
    but still symmetric) group law $(x_1, y_1) + (x_2, y_2) = (x_3, y_3)$ where
\begin{align}\label{eq:binary_add}
x_3 &=  \frac{d_1(x_1 + x_2) + d_2(x_1 + y_1)(x_2 + y_2) + (x_1 + x_1^2)
            (x_2(y_1 + y_2 + 1) + y_1y_2)}{d_1 + (x_1 + x_1^2)(x_2 + y_2)}\\
\notag
y_3 &=  \frac{d_1(y_1 + y_2) + d_2(x_1 + y_1)(x_2 + y_2) + (y_1 + y_1^2)
            (y_2(x_1 + x_2 + 1) + x_1x_2)}{d_1 + (y_1 + y_1^2)(x_2 + y_2)}
\end{align}

All of four of the normal forms we examine have group laws that are
    purported to be unified and complete (at least on a specified subgroup).
They fail to live up to the Edwards standard in other ways, however.
A few of these normal forms have group laws that are asymmetric; that is, the
    equations for adding two points $P$ and $Q$ involve their coordinates in
    such a fashion that it's not obvious that $P + Q$ is the same as $Q + P$,
    even though addition of two rational points on an elliptic curve is
    commutative.
None of the three major Edwards curve types---the original one put forward in
    \cite{edwards2007normal} and \cite{bernstein2007faster}, binary curves
    presented in \cite{bernstein2008binary}, or twisted curves from
    \cite{bernstein2008twisted}---exhibit this flaw.
All three of the Edwards group laws---Edwards curves in equation
    \ref{eq:edwards_add}, twisted Edwards curves in equation
    \ref{eq:twisted_add}, and binary Edwards curves in
    \ref{eq:binary_add}---are symmetric with respect to their inputs; one can
    clearly see that $(x_1, y_1) + (x_2, y_2)$ is the same as $(x_2, y_2) +
    (x_1, y_1)$ without any extra work simply because of the commutativity of
    field addition and multiplication.
This means that any implementation of these laws in computer code will be much
    less complex than they otherwise could be if extra work were needed to
    demonstrate this simple fact.

The other weakness exhibited by some of the normal forms we examine is their
    atypical choice of neutral element.
For some, the neutral element choice makes it unclear that $\mathcal{O} + P = P
    + \mathcal{O} = P$.
For Edwards curves, the neutral element is $(0, 1)$.
It's clear that this can be substituted into the Edwards group law in either
    position and the result will always be the other point; that is, it's
    immediately clear that $(0, 1)$ is indeed the neutral element for this law.
Similarly, twisted Edwards curves have neutral point $(0, 1)$, while binary
    Edwards curves have neutral point $(0, 0)$.
Substituting these into either position in their group laws clearly
    demonstrates that they are the correct neutral elements.
For some of the variations, it is not so apparent that
    the stated neutral element is correct; we again need to resort to reducing
    modulo the ideal generated by the curve equation in order to see that this
    is the case.


\bodysection{Farashahi \& Joye}

The first curve we'll consider is Farashahi and Joye's Generalized Hessian
    curve presented in \cite{farashahi2010efficient}.
This curve has the form
\[
H_{c, d} : x^3 + y^3 + c = dxy
\]
    or, in projective coordinates, 
\[
\mathbf{H}_{c, d} : X^3 + Y^3 + cZ^3 = dXYZ
\]
    over an arbitrary field.
The group of rational points on this curve has neutral element $\mathcal{O} =
    (1 : -1 : 0)$.

The authors present some unified addition formulas for $\mathbf{H}_{c, d}$
    (equations (9) and (10) in \cite{farashahi2010efficient}).
If we let $P = (X_1 : Y_1 : Z_1)$ and $Q = (X_2 : Y_2 : Z_2)$ be two points on
    $\mathbf{H}_{c, d}$, then according to their first equation we have $P + Q
    = (X_3 : Y_3 : Z_3)$ where
\begin{align*}
X_3 &=  cY_2Z_1^2Z_2 - X_1X_2^2Y_1\\
Y_3 &=  X_2Y_1^2Y_2 - cX_1Z_1Z_2^2\\
Z_3 &=  X_1^2X_2Z_2 - Y_1Y_2^2Z_1
\end{align*}
Using these formulas, we can calculate $\mathcal{O} + P = (X_1^2 : X_1Y_1 :
    X_1Z_1)$; while at first this may not seem to be the same as $P$,
    projective points are really equivalence classes, so this is of course the
    same point as we would get dividing all three positions by
    $X_1$,\footnote{$X_1$ cannot be zero, or else $\mathcal{O} + P$ would be a
    singular point on $\mathbf{H}_{c, d}$, something which the authors show is
    impossible.} viz.  $(X_1 : Y_1 : Z_1) = P$ provided, of course, that $X_1
    \ne 0$.
Similarly, $P + \mathcal{O} = (-X_1Y_1 : -Y_1^2 : -Y_1Z_1) \equiv
    (X_1 : Y_1 : Z_1) = P$.

The real trouble with this construction, however, comes from comparing $P + Q$
    with $Q + P$.
Let $(X_4 : Y_4 : Z_4) = Q + P$, so
\begin{align*}
X_4 &=  cY_1Z_1Z_2^2 - X_1^2X_2Y_2\\
Y_4 &=  X_1Y_1Y_2^2 - cX_2Z_1^2Z_2\\
Z_4 &=  X_1X_2^2Z_1 - Y_1^2Y_2Z_2
\end{align*}
Since we need point addition to be commutative, this should be equal (or at
    least equivalent in the projective point sense) to $P + Q$.
Suppose that all of $P, Q, P + Q$, and $Q + P$ are finite points, so their $Z$
    coordinate is nonzero.
Then we need the following:
\[
\frac{X_3}{Z_3} = \frac{cY_2Z_1^2Z_2 - X_1X_2^2Y_1}{X_1^2X_2Z_2 - Y_1Y_2^2Z_1}
=
\frac{cY_1Z_1Z_2^2 - X_1^2X_2Y_2}{X_1X_2^2Z_1 - Y_1^2Y_2Z_2} = \frac{X_4}{Z_4}
\]
This is true if and only if $X_3Z_4 - X_4Z_3 = 0$; i.e. if and only if the
    following is zero:
\begin{equation}\label{f_j_add}
-X_1X_2\left(cX_1Y_1Z_1Z_2^3 - cX_2Y_2Z_2Z_1^3 - X_1^3X_2Y_2Z_2 +
    X_1Y_1Z_1X_2^3 + X_1Y_1Z_1Y_2^3 - X_2Y_2Z_2^3\right)
\end{equation}
Suppose furthermore that $X_1X_2 \ne 0$; then we need the larger factor to be
    zero, which isn't immediately apparent.
Factoring and simplifying, this larger factor becomes
\[
(X_1Y_1Z_1)(X_2^3 + Y_2^3 + cZ_2^3) - (X_2Y_2Z_2)(X_1^3 + Y_2^3 + cZ_1^3)
\]
Working modulo the curve equation, we know $X^3 + Y^3 + cZ^3 = dXYZ$, which
    implies our work simplifies to
\[
(X_1Y_1Z_1)(dX_2Y_2Z_2) - (X_2Y_2Z_2)(dX_1Y_1Z_1)
\]
    which is, at last, zero.

Similarly,
\begin{align*}
&Y_3Z_4 - Y_4Z_3 =\\
&(X_1^2X_2Z_3 - Y_1Y_2^2Z_1)(cX_2Z_1^2Z_2 - X_1Y_1Y_2^2) -
    (X_1X_2^2Z_1 - Y_1^2Y_2Z_2)(cX_1Z_1Z_2^2 - X_2Y_1^2Y_2) =\\
&Y_1Y_2(cX_1Y_1Z_1Z_2^3 - cX_2Y_2Z_1^3Z_2 + X_1X_2^3Y_1Z_1 - X_1^3X_2Y_2Z_2 +
    X_1Y_1Y_2^3Z_1 - X_2Y_1^3Y_2Z_2) =\\
&Y_1Y_2\left[(X_1Y_1Z_1)(X_2^3 + Y_2^3 + cZ_2^3) - (X_2Y_2Z_2)(X_1^3 + Y_1^3 +
    cZ_1^3)\right]
\end{align*}
If $Y_1Y_2 \ne 0$, then this can only be zero if we resort to the curve
    equation, getting
\[
Y_1Y_2\left[(X_1Y_1Z_1)(dX_2Y_2Z_2) - (X_2Y_2Z_2)(dX_1Y_1Z_1)\right]
\]

Thus $P + Q$ does indeed equal $Q + P$; note, however, that in order to reach
    this conclusion we had to perform substitutions using $\mathbf{H}_{c, d}$'s
    equation.
This equality was not apparent from the outset but rather required working
    modulo the ideal generated by the curve equation in the appropriate
    polynomial ring.
This addition is true, and even mathematically pleasing, but not
    cryptographically viable.
Such reductions would complicate any computer code implementation of this
    group---at best leading to slow execution speed, and at worst causing
    side-channel leaks that could potentially lead to a break of the
    implementation.
This elliptic curve is not as safe as Edwards curves when it comes to the
    concerns of cryptographic implementation.

\bodysection{Wang, Tang, \& Yang}

In \cite{wang2012new}, the authors explore the curve
\[
M_d: x^2y + xy^2 + dxy + 1 = 0
\]
    and its homogeneous projective version
\[
\widetilde{M_d}: X^2Y + XY^2 + dXYZ + Z^3 = 0
\]
    over a field of characteristic greater than three.\footnote{Of course
    characteristic greater than three means that this curve is not a direct
    competitor to binary Edwards curves as such. However, it attempts to have
    a unified group law like Edwards curves do and fails for reasons similar
    to the other normal forms we analyze; these reasons make it worth including
    in our discussion.}
The neutral element of the group of rational points on this curve is
    $(1 : -1 : 0)$
Though their affine group law seems to have little trouble in the symmetry
    department, the projective group law is where the real trouble lies.
Per the law given in \cite{wang2012new}, the sum of two points $(X_1 : Y_1 :
    Z_1)$ and $(X_2 : Y_2 : Z_2)$ is $ (X_3 : Y_3 : Z_3)$ where
\begin{align*}
X_3 &=  X_1X_2(Y_1Z_2 - Y_2Z_1)^2\\
Y_3 &=  Y_1Y_2(X_1Z_2 - X_2Z_1)^2\\
Z_3 &=  (X_1Z_2 - X_2Z_1)(Y_1Z_2 - Y_2Z_1)(X_2Y_2Z_1^2 - X_1Y_1Z_2^2)
\end{align*}
    is problematic with regards to the neutral element.
Suppose we wished to add the point $P = (X : Y : Z)$ (a finite point, so $Z \ne
    0$) and the neutral element $(1 : -1 : 0)$; then we'd have
\begin{align*}
X_3 &=  X \cdot 1 (Y \cdot 0 - (-1) \cdot Z)^2\\
Y_3 &=  Y \cdot (-1) (X \cdot 0 - 1 \cdot Z)^2\\
Z_3 &=  (X \cdot 0 - 1 \cdot Z)(Y \cdot 0 - (-1) \cdot Z)
    (1 \cdot (-1) \cdot Z^2 - X \cdot Y \cdot 0^2)
\end{align*}
    which simplifies to $(XZ^2 : -YZ^2 : Z^4) \equiv (X : -Y : Z^2)$.
Except in very special circumstances, this is of course not equal to
    $(X : Y : Z)$; moreover, it's not apparent how resorting to the curve
    equation will even help here.

There are even more problems here, though.
For example, $\mathcal{O} + P = (XZ^2 : -YZ^2 : -Z^4) \equiv (X : -Y : -Z^2)$,
    $\mathcal{O} + \mathcal{O} = (0 : 0 : 0)$, and $P + P = (0 : 0 : 0)$, so
    this law is not unified (contrary to the claims of \cite{wang2012new}).
These problems can be seen by running the following Sage \cite{sage} script:
\lstinputlisting[caption={Arithmetic on Wang et.al.'s curve},language=Python]{listings/wang.sage}
Hence this curve is not a suitable candidate for cryptographic implementation.


\bodysection{Wu, Tang, \& Feng}

In \cite{wu2010new}, presented at INDOCRYPT 2012, Wu, Tang, \& Feng introduce
    the curve
\[
S_t: x^2y + xy^2 + txy + x + y = 0
\]
    and its projective version
\[
X^2Y + XY^2 + tXYZ + XZ^2 + YZ^2 = 0
\]
    and study its properties over a binary field.
In their paper, they define the projective point $\mathcal{O} = (1 : 1 : 0)$ as
    the neutral element.

Suppose that we wish to add the finite projective point $(X : Y : 1)$ to
    $\mathcal{O}$ using the formulas given in \cite{wu2010new} to obtain the
    point $(X_3 : Y_3 : Z_3)$; moreover, suppose that $X \ne Y$ and both are
    nonzero.
Then
\begin{align*}
X_3 &=  (Y \cdot 1 + 1 \cdot 0)\left[(X \cdot 1 + Y \cdot 1)
        (Y \cdot 0 + 1 \cdot 1) + t \cdot Y \cdot 1\cdot(1 \cdot 0 + X \cdot 1)
        \right]\\
    &=  X\left[(X + Y) + tXY\right]\\
    &=  X(X + Y + tXY)\\
Y_3 &=  (Y \cdot 1 + 1 \cdot 0)\left[(X \cdot 1 + Y \cdot 1)
        (X \cdot 0 + 1 \cdot 1) + t \cdot X \cdot 1 (1 \cdot 0 + Y \cdot 1)
        \right]\\
    &=  Y\left[(X + Y) + tXY\right]\\
    &=  Y(X + Y + tXY)\\
Z_3 &=  (X \cdot 1 + Y \cdot 1)(X \cdot 1 + 1 \cdot 0)(Y \cdot 1 + 1 \cdot 0)\\
    &=  XY(X + Y)
\end{align*}

Therefore $(X_3 : Y_3 : Z_3)$ is equivalent to
\begin{align*}
&\left(
\frac{X(X + Y + tXY)}{XY(X + Y)} : \frac{Y(X + Y + tXY)}{XY(X + Y)} : 1
\right)
=\\
&\left(
\frac{X + Y + tXY}{Y(X + Y)} : \frac{X + Y + tXY}{X(X + Y)} : 1
\right)
\end{align*}

From the curve equation, we know that $X + Y + tXY = X^2Y + XY^2 = XY(X + Y)$,
    so $(X_3 : Y_3 : Z_3)$ is indeed equal to $(X : Y : 1)$.
Note, however, that this result only occurs if we take into account the curve
    equation.
For something as simple as adding a point to the neutral element, having to
    modulo the curve equation to show that $(X : Y : 1) + \mathcal{O} =
    (X : Y : 1)$ is unnecessarily complicated.

\bodysection{Diao \& Fouotsa}

In \cite{diao2012edwards}, presented at ``Journ\'ees C2: Codage et
    Cryptographie'' in September 2012, Diao \& Fouotsa introduce the curve
\[
\mathcal{E}_\lambda: 1 + x^2 + y^2 + x^2y^2 = \lambda xy
\]
    which is valid over a field of any characteristic.
Their paper is very detailed, and the construction involves some interesting
    work with Theta functions.
Unfortunately, this construction also falls short of the cryptographic
    applicability of Edwards curves due to the asymmetry of the group law they
    present.

Suppose we wished to add two points $(x_1, y_1)$ and $(x_2, y_2)$; it shouldn't
    matter in which order we add them, because the group law should be
    commutative.
By the work in \cite{diao2012edwards}, we have
\[
(x_1, y_1) + (x_2, y_2)
= \left(
\frac{x_1 + y_1x_2y_2}{y_2 + x_1y_1x_2},
\frac{x_1x_2 + y_1y_2}{1 + x_1x_2y_1y_2}
\right)
\]
    while
\[
(x_2, y_2) + (x_1, y_1)
= \left(
\frac{x_2 + x_1y_1y_2}{y_1 + x_1x_2y_2},
\frac{x_1x_2 + y_1y_2}{1 + x_1x_2y_1y_2}
\right)
\]

The second coordinates of these points are obviously equal to each other, but
    we also need the first ones to be equal.
This is the case if and only if
\begin{align*}
&\frac{x_1 + y_1x_2y_2}{y_2 + x_1y_1x_2}
    = \frac{x_2 + x_1y_1y_2}{y_1 + x_1x_2y_2} \iff\\
&(x_1 + y_1x_2y_2)(y_1 + x_1x_2y_2)
    = (x_2 + x_1y_1y_2)(y_2 + x_1x_2y_1) \iff\\
&x_1y_1 + x_2(x_1^2 + y_1^2 + x_1y_1x_2y_2)
    = x_2y_2 + x_1y_1(x_2^2 + y_2^2 + x_1x_2y_1y_2)
\end{align*}

Using the curve equation this is true if and only if
\begin{align*}
&x_1y_1 + x_2y_2(1 + x_1^2y_1^2 + x_1x_2y_1y_2)
    = x_2y_2 + x_1y_1(1 + x_2^2y_2^2 + x_1x_2y_1y_2) \iff\\
&x_1y_1 + x_2y_2 + x_1^2x_2y_1^2y_2 + x_1x_2^2y_1y_2^2
    = x_2y_2 + x_1y_1 + x_1x_2^2y_1y_2^2 + x_1^2x_2y_1^2y_2
\end{align*}
So it is true that $(x_1, y_1) + (x_2, y_2) = (x_2, y_2) + (x_1, y_1)$ as we
    required.
Note that proving this simple fact again required resorting to working modulo
    the curve equation (i.e. modulo the ideal generated by the curve equation
    in the polynomial ring $\mathbb{F}_2^n[x_1, x_2, y_1, y_2]$).

\bodysection{Conclusions}

Following the excitement regarding the various types of Edwards curves,
    normal forms for elliptic curves have been presented and explored with an
    eye to improving upon one characteristic or another of Edwards curves while
    maintaining the same safety and security afforded by their complete and
    unified group laws.
It turns out that there is more to being as safe as Edwards curves than just
    being complete (on a subgroup or over the whole group) and unified,
    however.
As we have demonstrated, four recently proposed normal forms exhibit weaknesses
    that don't show up in Edwards curves: either their group laws are not
    symmetric or they use an unusual choice of neutral element.\footnote{In
    fact, one normal form's troubles extend even deeper.}
Both of these weaknesses mean that we must reduce modulo their curve equations
    to demonstrate even elementary facts, like $\mathcal{O} + P = P$ or $P + Q
    = Q + P$.
This extra work will complicate any computer implementation, leading to slower
    execution speed and perhaps leakage of information through side channels.
The main advantage Edwards curves have for implementation is their
    incorporating safety and security from the ground up; these newer normal
    forms do not measure up when it comes to suitability for cryptographic
    implementation.
