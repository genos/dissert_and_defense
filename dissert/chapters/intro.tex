\bodychapter{Introduction}
\label{chp:intro}

The group of rational points on an elliptic curve over a finite field has
    proven very useful in cryptography since Miller and Koblitz first suggested
    its use independently in the 1980s (\cite{miller1986use} and
    \cite{koblitz1987elliptic}).
Due to the lack of subexponential algorithms to solve the Discrete Logarithm
    Problem in this group, elliptic curve cryptography cryptosystems tend have
    to have a level of security comparable to other ElGamal-type systems (e.g.
    ones that use the DLP, or more precisely the Diffie Hellman problem, over
    the multiplicative group $\mathbb{F}_{p}^\ast$; see
    \cite{stinson2005cryptography}) while using much smaller key sizes.
Moreover, the group of points on an elliptic curve $E$ over a finite field $K$
    is rather nice to work with; it's isomorphic to either a cyclic group or
    the direct product of two cyclic groups and (at least in the typical
    Weierstrass coordinates) has a simple geometric interpretation.

There is some room for improvement, however.
Typically the group operation on $E$ involves a number of special cases, all of
    which must be checked for at every turn:
\begin{itemize}
\item   What if one point is the neutral element, the so-called ``point at
    infinity?''
\item   What if the two points are the same?
    What if their $x$-coordinate is zero?
\item   What if the two points are inverses of each other?
\end{itemize}
In each of these cases, the exception to the usual formula can cause
    implementations to giving up more information to outside observers than
    intended---leaking ``side channel information.''
That is, though the theory of elliptic curve cryptography is perfectly sound
    from a mathematical standpoint, in practice it is either less secure (or at
    least more complicated) than originally thought due to shortcomings in the
    theory's applicability.
Such types of attacks have been outlined in papers like
    \cite{brier2002weierstrass} and \cite{izu2002exceptional}, and considerable
    effort has been spent trying to make Weierstrass curve implementations
    secure ``after the fact,'' as it were; see e.g. \cite{moller2001securing}.

In 2007, Dr. Harold Edwards discussed a new normal form for elliptic curves in
    \cite{edwards2007normal}.
Despite his paper not focusing on cryptography, the normal form put forth by
    Edwards has very desirable cryptographic properties that help combat the
    leakage of side-channel information from the very start; as noted by
    Bernstein and Lange in \cite{bernstein2007faster}, the group law is
    \textit{complete} and \textit{unified}, two terms we will discuss later.
Moreover, in many cases the group law involves less operations, meaning that
    the more secure computations involved can also be faster.
While this is not the case over binary fields, the benefits of the law's
    completeness make the loss of speed seem negligible; in fact,
    \cite{moloneyefficient}'s authors argue that with specialized hardware
    the speed difference can be greatly reduced, while the completeness of the
    binary Edwards curve group law actually makes it faster than Weierstrass
    implementations that must constantly check for special cases.
Add to this the reduced complexity of implementation code, and binary Edwards
    are just as competitive as their non-binary counterparts.

Though there has been significant work to build up the literature on Edwards
    curves, there is still room to explore the cryptographic and mathematical
    aspects of these new normal forms.
In this dissertation we do just that, specifically focusing on binary Edwards
    curves (which have been explored less than other types).
Our work is as follows: in Chapter \ref{chp:back}, we begin with the necessary
    background on elliptic curves, elliptic curve cryptography, and two of the
    three types of Edwards curves.
Then in Chapter \ref{chp:bec}, we build on the theoretical foundation of the
    previous Chapter to discuss binary Edwards curves.
Next, in Chapter \ref{chp:prax}, we examine the practical considerations
    involved in applying that theory in cryptographic practice, including a
    discussion of the shortcomings of four recently proposed normal forms.
After that, we explore pairing computations over binary Edwards curves in
    \ref{chp:pair}.
In \ref{chp:app}, we focus on two applications of binary Edwards curves to
    cryptography: password-based key derivation functions and a compartmented
    secret sharing scheme with signcryption.
Finally, in Appendix \ref{chp:e2c2} we discuss \texttt{e2c2}, a modern computer
    software library written in C++11 to perform Edwards elliptic curve
    cryptography built on top of Shoup's \texttt{NTL} \cite{shoup2009ntl}; the
    sourcecode for \texttt{e2c2} is included in appendix \ref{chp:src}.
