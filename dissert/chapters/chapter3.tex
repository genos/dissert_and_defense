\bodychapter{Binary Edwards Curves}
\label{chp:bec}

In this chapter we discuss binary Edwards curves.
We'll start with a discussion of the work of \cite{bernstein2008binary}, the
    first paper to lay out an ``Edwards-like'' elliptic curve over a field of
    characteristic two.
Then, we'll look at the practical improvements provided by
    \cite{moloneyefficient}.

Bernstein, Lange, and Farashahi's paper \cite{bernstein2008binary} presented
    ``the first complete addition formulas for binary elliptic curves.''
As such, it was a huge milestone in the field; binary elliptic curves are very
    attractive from an implementation standpoint because, after all, computers
    work in binary.
Until this paper came along, ECC implementations over a binary finite field
    were inherently vulnerable to the types of side-channel attacks mentioned
    in the previous chapter.
Moloney, O'Mahony, and Laurent's paper \cite{moloneyefficient} extended this
    work, presenting algorithms and practical measurements of things like code
    complexity that matter to implementors of cryptographic primitives.

\bodysection{Bernstein, Lange, \& Farashahi}

Unfortunately for cryptographers, the Edwards curve equation $x^2 + y^2 = c^2(1
    + dx^2y^2)$ is not elliptic over fields with characteristic two; if it
    were, one could just use Edwards curves (or twisted Edwards curves) over
    these fields and reap the same benefits that we did over non-binary fields.
In 2008, Bernstein, Lange, and Farashahi came up with a normal form for
    elliptic curves over binary fields that is reminiscent of Edwards curves
    which they dubbed \textit{binary Edwards curves}.

\begin{dfn}
Let $K$ be a field with $char(K) = 2$, and $d_1, d_2 \in K$ such that
    $d_1 \ne 0$ and $d_2 \ne d_1^2 + d_1$.
The \textit{binary Edwards curve} $E_{B, d_1, d_2}$ is the affine curve
\[
d_1(x + y) + d_2(x^2 + y^2) = xy(x + 1)(y + 1)
\]
\end{dfn}

As you can see from the definition, $E_{B, d_1, d_2}$ is symmetric in $x$ and
    $y$, so if $(x, y)$ is a point on $E_{B, d_1, d_2}$ then so is $(y, x)$;
    this will soon yield our negation law.
There are only two points on the curve that are invariant under this law: $(0,
    0)$ and $(1, 1)$.
As we'll see shortly, the former will be our neutral element, while the latter
    will be a point of order two.

In their theorem (2.2), the authors of \cite{bernstein2008binary} show that
    the affine form of $E_{B, d_1, d_2}$ is nonsingular.
Shortly after, they look at singularities of the projective closure
\[
d_1(X + Y)Z^3 + d_2(X^2 + Y^2)Z^2 = XY(X + Z)(Y + Z)
\]
    of which there are two: $\Omega_1 = (1 : 0 : 0)$ and $\Omega_2 = (0 : 1 :
    0)$.
We'll expand on their work to show that the first of these blows up to two
    projective points, and use their same appeal to symmetry to cover the
    second.

To study $E_{B, d_1, d_2}$ around $\Omega_1$, consider the affine curve
    $E_{\Omega_1}$:
\[
d_1(1 + y)z^3 + d_2(1 + y^2)z^2 + y(1 + z)(y + z) = 0
\]
If we take the partial derivatives of this curve with respect to $y$ and $z$,
    we get
\begin{align*}
\frac{\partial E_{\Omega_1}}{\partial y}
    &=  d_1z^3 + 2d_2yz^2 + (z + 1)(y + z) + (z + 1)y\\
    &=  d_1z^3 + 2d_2yz^2 + 2yz + z^2 + 2y + z\\
    &=  d_1z^3 + z^2 + z
\end{align*}
    and
\begin{align*}
\frac{\partial E_{\Omega_1}}{\partial z}
    &=  3(y + 1)d_1z^2 + 2(y^2 + 1)d_2z + (z + 1)y + (y + z)y\\
    &=  3d_1yz^2 + 2d_2y^2z + 3d_1z^2 + 2d_2z + y^2 + 2yz + y\\
    &=  d_1(1 + y)z^2 + y^2 + y
\end{align*}
Evaluating these at the point $(y, z) = (0, 0)$, we see that $E_{\Omega_1}$ is
    indeed singular; we can ``blow up'' this singularity by substituting $y =
    tz$ into $E_{\Omega_1}$ and dividing through by $z^2$, getting the
    following curve $E_t$:
\[
d_1(1 + tz)z + d_2(1 + t^2z^2) + t(1 + t)(1 + z) = 0
\]
If we substitute in $z = 0$, $E_t$ becomes $t^2 + t + d_2 = 0$ which has two
    distinct roots in $\overline{K}$.
To see that these two points are nonsingular, consider the partial derivatives
\begin{align*}
\frac{\partial E_t}{\partial t}
    &=  2d_2tz^2 + d_1z^2 + (z + 1)(t + 1) + (z + 1)t\\
    &=  2d_2tz^2 + d_1z^2 + 2tz + 2t + z + 1\\
    &=  d_1z^2 + z + 1
\end{align*}
    and
\begin{align*}
\frac{\partial E_t}{\partial z}
    &=  2d_2t^2z + d_1tz + (t + 1)t + (tz + 1)d_1\\
    &=  2d_2t^2z + 2d_1tz + t^2 + d_1 + t\\
    &=  t^2 + d_1 + t
\end{align*}
Neither of these partial derivatives vanish at the point $(z, t) = (0, 0)$, so
    these blowups are nonsingular.
As \cite{bernstein2008binary} says, they are ``defined over the smallest
    extension of $K$ in which $d_2 + t + t^2 = 0$ has roots.''

The authors provide the following birational equivalence: the map
\begin{align*}
(x, y)
    &\mapsto    (u, v)\\
    &=  \left(
            \frac{d_1(d_1^2 + d_1 + d_2)(x + y)}{xy + d_1(x + y)}
            ,
            d_1(d_1^2 + d_1 + d_2)\left[
                \frac{x}{xy + d_1(x + y)} + d_1 + 1
            \right]
        \right)
\end{align*}
    is a birational equivalence\footnote{Though we'll use the equivalence given
    in \cite{moloneyefficient} ourselves.} between $E_{B, d_1, d_2}$ and the
    binary elliptic curve $W$\footnote{In shorter Weierstrass form for binary
    curves}
\[
v^2 + uv = u^3 + (d_1^2 + d_2)u^2 + d_1^4(d_1^4 + d_1^2 + d_2^2)
\]
This map has inverse
\[
(u, v) \mapsto (x, y)
    =   \left(
            \frac{d_1(u + d_1^2 + d_1 + d_2)}
                {u + v + (d_1^2 + d_1)(d_1^2 + d_1 + d_2)}
            ,
            \frac{d_1(u + d_1^2 + d_1 + d_2)}
                {v + (d_1^2 + d_1)(d_1^2 + d_1 + d_2)}
        \right)
\]
This map is undefined at the point $(0, 0)$; if we define $(0, 0) \mapsto
    \infty$, then this becomes an isomorphism between the curves.

The addition law on a binary Edwards curve is just as symmetric as its ordinary
    and twisted counterparts, if a little more complicated:

\begin{thm}[Binary Edwards Addition Law]
If $(x_1, y_1)$ and $(x_2, y_2)$ are two points on the binary Edwards curve
    $E_{B, d_1, d_2}$, then the mapping $(x_1, y_1), (x_2, y_2) \mapsto (x_3,
    y_3)$ turns the rational points on this curve into an abelian group, where
\begin{align*}
x_3 &=  \frac{d_1(x_1 + x_2) + d_2(x_1 + y_1)(x_2 + y_2) + (x_1 + x_1^2)
            (x_2(y_1 + y_2 + 1) + y_1y_2)}{d_1 + (x_1 + x_1^2)(x_2 + y_2)}\\
y_3 &=  \frac{d_1(y_1 + y_2) + d_2(x_1 + y_1)(x_2 + y_2) + (y_1 + y_1^2)
            (y_2(x_1 + x_2 + 1) + x_1x_2)}{d_1 + (y_1 + y_1^2)(x_2 + y_2)}
\end{align*}
    as long as the denominators in the above fractions are nonzero.
\end{thm}
Substituting $(0, 0)$ for either $(x_1, y_1)$ or $(x_2, y_2)$ in the above law,
    we see that $(0, 0)$ is the neutral element.
Moreover, $(x, y) + (y, x) = (0, 0)$, so the inverse of a point $(x, y)$ is
    $(y, x)$ as we said before.
When defined, this addition law is unified; it can be used for doubling as
    well.
For a proof of this law, see section 3 of \cite{bernstein2008binary}; in that
    section, the authors demonstrate that this addition law corresponds to the
    addition law on the equivalent Weierstrass curve, so the birational map is
    indeed a isomorphism.

Astute readers will notice the caveat ``as long as the denominators in the
    above fractions are nonzero'' in the previous theorem.
We could try and list all the cases where those fractions don't exist and piece
    together a group law that takes these special cases into account, like the
    Weierstrass group law does.
However, Bernstein, Lange, and Farashahi offer us another very helpful theorem.

\begin{thm}[Complete Binary Edwards Curves]
Let $K$ be a field with $char(K) = 2$ and $d_1, d_2 \in K$ such that $d_1 \ne
    0$ and no element $t \in K$ satisfies $t^2 + t + d_2 = 0$.
Then the addition law on the binary Edwards curve $E_{B, d_1, d_2}(K)$ is
    complete.
Moreover, every ordinary elliptic curve over the finite field
    $\mathbb{F}_{2^n}$ for $n \ge 3$ is birationally equivalent over
    $\mathbb{F}_{2^n}$ to a complete binary Edwards curve.
\end{thm}
See theorems (4.1) and (4.3) for proofs of these claims.
Since elliptic curve cryptography typically involves finite binary fields of
    degree $n$ at least $160$, the above theorem tells us that we can use a
    binary Edwards curve and reap the benefits of a complete and unified group
    law.

Despite their extremely thorough treatment in \cite{bernstein2008binary},
    Bernstein, Lange, and Farashahi did leave some small room for improvement.
In trying to find an equivalent complete binary Edwards curve for a given
    Weierstrass curve, they left some nondeterminism in finding $d_1$.
Though they could find an appropriate $d_1$ easily enough experimentally, they
    didn't have a  deterministic algorithm for it.

\bodysection{Moloney, O'Mahony, \& Laurent}

In 2010, Moloney, O'Mahony, and Laurent posted \cite{moloneyefficient} online.
In it, they perform a practical, implementation-focused analysis of binary
    Edwards curves, and come up with some very useful results.

First, they offer a modified birational equivalence.
Recall the usual trace function
\[
Tr: \mathbb{F}_{2^n} \to \mathbb{F}_2, \qquad
    \alpha \mapsto \sum_{i = 0}^{n - 1} \alpha^{2^i}
\]
    and define the half-trace function
\[
H: \mathbb{F}_{2^n} \to \mathbb{F}_2, \qquad
    \alpha \mapsto \sum_{i = 0}^{\sfrac{(n - 1)}{2}} \alpha^{2^{2^i}}
\]
    (noting that $n$ must be odd).
If given $a_2$ and $a_6$ for a Weierstrass curve, suppose we found a suitable
    $d_1$; we can then calculate
\[
d_2 = d_1^2 + d_1 + \sfrac{\sqrt{a_6}}{d_1^2}
\]
We will also make use of $b$ which satisfies $b^2 + b = d_1^2 + d_2 + a_2$; it
    can be directly calculated as
\[
b = H(d_1^2 + d_2 + a_2)
\]
The authors show that $(u, v) \mapsto (x, y)$ is another birational equivalence
    from the Weierstrass curve to $E_{B, d_1, d_2}$, where
\begin{align*}
x   &=  \frac{d_1(bu + v + (d_1^2 + d_1)(d_1^2 + d_1 + d_2))}
        {u^2 + d_1u + d_1^2(d_1^2 + d_1 + d_2)}\\
y   &=  \frac{d_1((b + 1)u + v + (d_1^2 + d_1)(d_1^2 + d_1 + d_2))}
        {u^2 + d_1u + d_1^2(d_1^2 + d_1 + d_2)}
\end{align*}
Though there is no difference between this equivalence and the one presented by
    Bernstein et.al., the calculation of this equivalence involves fewer field
    inversions.
Field inversions tend to be very costly to calculate, so the fewer the
    better.\footnote{\texttt{e2c2} uses this birational equivalence.}

Secondly, and perhaps more importantly, the authors present two deterministic
    algorithms to find a suitable $d_1$ given $n \ge 3$, $a_2$, and $a_6$
    determining a Weierstrass curve over the finite field $\mathbb{F}_{2^n}$.
We reproduce the first algorithm here, since that's what is used in our
    software library \texttt{e2c2}.
Precompute $t = Tr(a_2)$, $r = Tr(a_6)$, and $w = x + Tr(x)$ where $x$ is the
    indeterminant used to define our field extension $\mathbb{F}_{2^n}$.
This algorithm ``terminates with guaranteed success in a finite number of
    steps, except in the case $t = r = 0$.''
Fortunately, ``this case does not appear in any of the standards (e.g. NIST) of
    which the authors are aware.''

\begin{Algorithm}
\caption{Moloney, O'Mahony, \& Laurent's first $d_1$ finder}
\label{alg:d1}
\begin{algorithmic}
    \Function{MOLalg1}{$n, p, t, r, a_6, w$}
        \If{$t = 0$ and $r = 1$}
            \State $d_1 \gets 1$
        \Else
            \If{$t = 1$ and $r = 0$}
                \State $d_1 \gets \sqrt[4]{a_6}$
            \Else
                \If{$t = r = 1$ and $a_6 \ne 1$}
                    \If{$Tr(\sfrac{1}{a_6 + 1}) = 1$}
                        \State $d_1 \gets \sqrt{a_6} + \sqrt[4]{a_6}$
                    \Else
                        \State $d_1 \gets \sqrt[4]{a_6} + 1$
                    \EndIf
                \Else
                    \If{$t = 1$ and $a_6 = 1$}
                        \If{$Tr(\sfrac{1}{w}) = 1$}
                            \State $d_1 \gets w$
                        \Else
                            \If{$Tr(\sfrac{1}{(w + 1)}) = 0$}
                                \State $d_1 \gets \sfrac{1}{(w + 1)}$
                            \Else
                                \State $d_1 \gets 1 = \sfrac{1}{(w + 1)}$
                            \EndIf
                        \EndIf
                    \Else
                        \If{$t = r = 0$}
                            \If{$Tr(\sfrac{1}{(a_6 + 1)}) = 0$}
                                \State $d_1 \gets \sqrt[4]{a_6} + 1$
                            \Else
                                \State $i \gets 1$
                                \State $s \gets \sqrt{a_6}$
                                \While{$Tr(a_6^{2^i + 1}) = 0$}
                                    \State $s \gets s^2$
                                    \State $i \gets i + 1$
                                \EndWhile
                                \State $d_1 \gets \sfrac{1}{(s + 1)}$
                            \EndIf
                        \EndIf
                    \EndIf
                \EndIf
            \EndIf
        \EndIf
        \State \Return $d_1$
    \EndFunction
\end{algorithmic}
\end{Algorithm}

Finally, \cite{moloneyefficient} offers some valuable measurements and
    comparisons between implementations of Weierstrass and binary Edwards
    curves.
They note that ``implementing ECC from the textbooks leaves us with incredibly
    complex code,'' while implementations of binary Edwards curves have lower
    complexity.
The symmetric, unified, and complete group law takes a lot of the burden off of
    potential developers and implementors.
More interestingly, despite the larger operation count for the binary Edwards
    addition law, the fact Weierstrass implementations must constantly check
    for special cases slows them down considerably.
The cost measurements commonly mentioned in the literature ``do not take into
account the cost of checking'' if an operation ``is attempting to double the
    point at infinity,'' for example.
Moreover, ``performance is significantly different if implemented on a
    different processor.''
Integrating binary Edwards code into an existing ECC library, they found on one
    processor that, as may be expected, the binary Edwards curve code was
    slower.
However, on a different processor that pipelined instructions, the
    implementation could take advantage of the fact that the binary Edwards
    curve addition law involves no conditionals; ``due to the fact that we do
    not have to break the pipeline with checks for the point of infinity,''
    along with some other, more esoteric technical work on the part of the
    authors, ``we are able to increase the performance of [binary Edwards
    curves] such that it is approximately $25\%$ faster than the equivalent
    Weierstrass version.''
